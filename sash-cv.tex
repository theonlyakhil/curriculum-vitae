%-------------------------
% Resume in Latex
% Author : Sashwat K
% Current user : Akhil A.B
%------------------------

\documentclass[letterpaper,11pt]{article}
\usepackage{latexsym}
\usepackage[empty]{fullpage}
\usepackage{titlesec}
\usepackage{marvosym}
\usepackage[usenames,dvipsnames]{color}
\usepackage{verbatim}
\usepackage{enumitem}
\usepackage[hidelinks]{hyperref}
\usepackage{fancyhdr}
\usepackage[english]{babel}
\pagestyle{fancy}
\fancyhf{} 

%-------------------------
% clear all header and footer fields
%-------------------------

\fancyfoot{}
\renewcommand{\headrulewidth}{0pt}
\renewcommand{\footrulewidth}{0pt}

%-------------------------
% Adjust margins
%-------------------------

\addtolength{\oddsidemargin}{-0.5in}
\addtolength{\evensidemargin}{-0.5in}
\addtolength{\textwidth}{1in}
\addtolength{\topmargin}{-.5in}
\addtolength{\textheight}{1.0in}
\urlstyle{same}
\raggedbottom
\raggedright
\setlength{\tabcolsep}{0in}

%-------------------------
% Sections formatting
%-------------------------

\titleformat{\section}{
  \vspace{-4pt}\scshape\raggedright\large
}{}{0em}{}[\color{black}\titlerule \vspace{-5pt}]

%-------------------------
% Custom commands
%-------------------------

\newcommand{\resumeItem}[2]{
  \item\small{
    \textbf{#1}{: #2 \vspace{-2pt}}
  }
}

\newcommand{\resumeSubheading}[4]{
  \vspace{-1pt}\item
    \begin{tabular*}{0.97\textwidth}[t]{l@{\extracolsep{\fill}}r}
      \textbf{#1} & #2 \\
      \textit{\small#3} & \textit{\small #4} \\
    \end{tabular*}\vspace{-5pt}
}

\newcommand{\resumeSubItem}[2]{\resumeItem{#1}{#2}\vspace{-4pt}}
\renewcommand{\labelitemii}{$\circ$}
\newcommand{\resumeSubHeadingListStart}{\begin{itemize}[leftmargin=*]}
\newcommand{\resumeSubHeadingListEnd}{\end{itemize}}
\newcommand{\resumeItemListStart}{\begin{itemize}}
\newcommand{\resumeItemListEnd}{\end{itemize}\vspace{-5pt}}


%%%%%%  CV STARTS HERE  %%%%%%%%%%%%%%%%%%%%%%%%%%%%


\begin{document}

%----------HEADING-----------------
\begin{tabular*}{\textwidth}{l@{\extracolsep{\fill}}r}
  \textbf{\href{https://theonlyakhil.github.io/}{\Large Akhil A B}} & Email : \href{mailto:akhilnanosoft79@gmail.com}{akhilnanosoft79@gmail.com}\\
  \href{https://theonlyakhil.github.io}{https://sashuu6.github.io} & Mobile : +91 8281962557 \\
\end{tabular*}

%-----------EDUCATION-----------------
\section{Education}
  \resumeSubHeadingListStart
    \resumeSubheading
      {College of Engineering Trivandrum (CET)}{Thiruvananthapuram, India}
      {Post Graduation; Masters in Computer Application}{August. 2017 -- April. 2020}
    \resumeSubheading
      {Christ Nagar College}{Thiruvananthapuram, India}
      {Graduation; Bachelor in Computer Application}{June. 2014 -- May. 2017}
    \resumeSubheading
      {Kendriya Vidyalaya AirForce Station Akkulam}{Thiruvananthapuram, India}
      {High School; Computer Science}{April. 2012 -- March. 2014}
  \resumeSubHeadingListEnd

%-----------EXPERIENCE-----------------
\section{Experience}
  \resumeSubHeadingListStart
  \resumeSubheading
      {TechWithUs PVT LTD}{CET, Thiruvananthapuram}
      {Software Developer}{March 2019 - Present}
      \resumeItemListStart
        \resumeItem{ExamMarker}
          {ExamMarker is a product developed using python, electron, javascript, Web and PHP. I contributed in developing the sections that used python, electron and javascript.}
      \resumeItemListEnd
    \resumeSubheading
      {IoT Lab}{CET, Thiruvananthapuram}
      {Student faculty}{September 2017 - January 2019}
      \resumeItemListStart
        \resumeItem{Arduino UNO}
          {The Arduino UNO is an open-source micro-controller board based on the Microchip ATmega328P micro-controller and developed by Arduino.cc. As a member of student faculty, I was given the chance to take classes on Arduino UNO to the students of CET.}
        \resumeItem{Raspberry Pi}
          {The Raspberry Pi is a tiny and affordable computer that can be used for practical projects and to learn programming. I took workshops on Raspberry Pi.}
      \resumeItemListEnd
    \resumeSubheading
      {Appfabs PVT LTD}{Technopark, Thiruvananthapuram, India}
      {Intern}{May 2018 - September 2018}
      \resumeItemListStart
        \resumeItem{Cyber security}
          {Cybersecurity is the protection of internet-connected systems, including hardware, software and data, from cyberattacks. In a computing context, security comprises cybersecurity and physical security -- both are used by enterprises to protect against unauthorised access to data centres and other computerised systems.}
      \resumeItemListEnd
  \resumeSubHeadingListEnd
  
%--------PROGRAMMING SKILLS------------
\section{Programming Skills}
  \resumeSubHeadingListStart
    \item{
      \textbf{Languages}{: Embedded C, Python, Javascript, Bash Script }
      \hfill
      \textbf{Technologies}{: Git, Linux, Docker, Electron }
    }
  \resumeSubHeadingListEnd

%-----------Workshops Taken-----------------
\section{Workshops Taken}
  \resumeSubHeadingListStart
    \resumeSubheading
      {Raspberry Pi Workshop}{CET, Thiruvananthapuram}
      {IOT Lab}{2018}
      \resumeItemListStart
        \resumeItem{Python}
          {This workshop was conducted by a team of 4 students including me. In this workshop, I handled python session.}
      \resumeItemListEnd
      \resumeSubheading
      {Arduino Workshop}{CET, Thiruvananthapuram}
      {IOT Lab}{2018}
      \resumeItemListStart
        \resumeItem{Arduino}
          {This workshop covered all the sections from Arduino basics, embedded C, micro controller vs micro processor and few basic programs.}
      \resumeItemListEnd
      
      \resumeSubheading
      {Beaglebone Workshop}{CET, Thiruvananthapuram}
      {IOT Lab}{2018}
      \resumeItemListStart
        \resumeItem{Linux and Python}
          {This workshop was conducted by a team of 4 students including me. In this workshop, I handled Linux and Python session.}
      \resumeItemListEnd
      
      \resumeSubheading
      {Raspberry Pi Workshop}{CET, Thiruvananthapuram}
      {Drishti}{2018}
      \resumeItemListStart
        \resumeItem{Pi}
          {This workshop was conducted at Drishti (Technical Fest, CET). The session included about introduction Raspberry Pi, Linux, working remotely, Python, creating Pi server and home automation using Pi.}
      \resumeItemListEnd
      
      \resumeSubheading
      {Git Workshop}{Muthoot Institute of Technology and Science, Kochi}
      {}{2019}
      \resumeItemListStart
        \resumeItem{Git}
          {In this workshop, I took session on Git and Github.}
      \resumeItemListEnd
      
  \resumeSubHeadingListEnd

%-----------CONTRIBUTED PROJECTS-----------------
\section{Contributed Projects}
  \resumeSubHeadingListStart
    \resumeSubItem{Python Fingerprint Recognition}
      { In this project, I have fixed the python errors due to which the program doesn't run. Parent project-\url{https://github.com/kjanko/python-fingerprint-recognition}. My project-\url{https://github.com/sashuu6/python-fingerprint-recognition}.}
  \resumeSubHeadingListEnd

%-----------MY PROJECTS-----------------
\section{My Projects}
  \resumeSubHeadingListStart
    \resumeSubItem{3-factor-authenticated-door-lock}
      { A 3 factor authenticated door lock using Atmega328p, 74HC595 multiplexer, key switch, keypad and RFID. This project is available at \url{https://github.com/sashuu6/3-factor-authenticated-door-lock}.}
    \resumeSubItem{digital-clock-with-birthday-alarm}
      {The digital clock with birthday alarm is a DYI alarm built using Arduino UNO, Adafruit OLED and DS 3231 RTC module. This project is available at \url{https://github.com/sashuu6/digital-clock-with-birthday-alarm}.}
    \resumeSubItem{youtube-sub-count}
      {The 'Youtube Subscriber and View Counter' is a device made using NodeMCU (ESP8266) and OLED display. You can use this device to view your youtube's subscriber and view count. This project is available at \url{https://github.com/sashuu6/youtube-sub-count}.}
    \resumeSubItem{simple-home-automation}
      {A simple home automation using Raspberry Pi Zero W, Particle.io, Google assistant and IFTTT. This project is available at \url{https://github.com/sashuu6/simple-home-automation}.}
      \resumeSubItem{vOne}
      {My version of Arduino Uno for board isolation. This project is available at \url{https://github.com/sashuu6/vOne}.}
       \resumeSubItem{Realtime Barcode Scanner}
      {My version of Barcode Scanner developed using Python and OpenCV. This project is available at \url{https://github.com/sashuu6/realtime-barcode-scanner}.}
  \resumeSubHeadingListEnd

%-------------------------------------------
\end{document}
