%-------------------------
% Resume in Latex
% Author : Akhil A B
% Current user : Akhil A.B
%------------------------

\documentclass[letterpaper,11pt]{article}
\usepackage{latexsym}
\usepackage[empty]{fullpage}
\usepackage{titlesec}
\usepackage{marvosym}
\usepackage[usenames,dvipsnames]{color}
\usepackage{verbatim}
\usepackage{enumitem}
\usepackage[hidelinks]{hyperref}
\usepackage{fancyhdr}
\usepackage[english]{babel}
\pagestyle{fancy}
\fancyhf{} 

%-------------------------
% clear all header and footer fields
%-------------------------

\fancyfoot{}
\renewcommand{\headrulewidth}{0pt}
\renewcommand{\footrulewidth}{0pt}

%-------------------------
% Adjust margins
%-------------------------

\addtolength{\oddsidemargin}{-0.5in}
\addtolength{\evensidemargin}{-0.5in}
\addtolength{\textwidth}{1in}
\addtolength{\topmargin}{-.5in}
\addtolength{\textheight}{1.0in}
\urlstyle{same}
\raggedbottom
\raggedright
\setlength{\tabcolsep}{0in}

%-------------------------
% Sections formatting
%-------------------------

\titleformat{\section}{
  \vspace{-4pt}\scshape\raggedright\large
}{}{0em}{}[\color{black}\titlerule \vspace{-5pt}]

%-------------------------
% Custom commands
%-------------------------

\newcommand{\resumeItem}[2]{
  \item\small{
    \textbf{#1}{: #2 \vspace{-2pt}}
  }
}

\newcommand{\resumeSubheading}[4]{
  \vspace{-1pt}\item
    \begin{tabular*}{0.97\textwidth}[t]{l@{\extracolsep{\fill}}r}
      \textbf{#1} & #2 \\
      \textit{\small#3} & \textit{\small #4} \\
    \end{tabular*}\vspace{-5pt}
}

\newcommand{\resumeSubItem}[2]{\resumeItem{#1}{#2}\vspace{-4pt}}
\renewcommand{\labelitemii}{$\circ$}
\newcommand{\resumeSubHeadingListStart}{\begin{itemize}[leftmargin=*]}
\newcommand{\resumeSubHeadingListEnd}{\end{itemize}}
\newcommand{\resumeItemListStart}{\begin{itemize}}
\newcommand{\resumeItemListEnd}{\end{itemize}\vspace{-5pt}}


%%%%%%  CV STARTS HERE  %%%%%%%%%%%%%%%%%%%%%%%%%%%%


\begin{document}

%----------HEADING-----------------
\begin{tabular*}{\textwidth}{l@{\extracolsep{\fill}}r}
  \textbf{\href{https://theonlyakhil.github.io/}{\Large Akhil A B}} & Email : \href{mailto:akhilab979@gmail.com}{akhilab979@gmail.com}\\
  \href{https://theonlyakhil.github.io}{https://theonlyakhil.github.io} & Mobile : +91 9074679826 \\
\end{tabular*}

%-----------EDUCATION-----------------
\section{Education}
  \resumeSubHeadingListStart
    \resumeSubheading
      {College of Engineering Trivandrum (CET)}{Thiruvananthapuram, India}
      {Graduation; B.Tech Applied Electronics and Instrumentation}{August. 2016 -- April. 2020}
    \resumeSubheading
      {Technical Higher Secondary School Muttada}{Thiruvananthapuram, India}
      {Higher Secondary; Electronics Service Technology; Score 78\%}{June. 2014 -- May. 2016}
  \resumeSubHeadingListEnd

%-----------EXPERIENCE-----------------
\section{Experience}
  \resumeSubHeadingListStart
  \resumeSubheading
      {TechWithUs PVT LTD}{CET, Thiruvananthapuram}
      {Hardware Developer}{March 2019 - Present}
      \resumeItemListStart
        \resumeItem{ExamMarker}
          {ExamMarker is a product developed using python, electron, javascript, Web and PHP. I contributed in developing the sections that used python for creating graphical user interface. I handled the hardware part of the device. The hardware includes power management, single board computer and many more.}
      \resumeItemListEnd
    \resumeSubheading
      {IoT Lab}{CET, Thiruvananthapuram}
      {Student faculty}{September 2016 - January 2019}
      \resumeItemListStart
        \resumeItem{Arduino UNO}
          {The Arduino UNO is an open-source micro-controller board based on the Microchip ATmega328P micro-controller and developed by Arduino.cc. As a member of student faculty, I was given the chance to take classes on Arduino UNO to the students of CET.}
        \resumeItem{Raspberry Pi}
          {The Raspberry Pi is a tiny and affordable computer that can be used for practical projects and to learn programming. I took workshops on Raspberry Pi.}
          \resumeItem{LoRa}
          {LoRa (Long Range) is a spread spectrum modulation technique. It is the first low-cost implementation of chirp spread spectrum for commercial usage.}
      \resumeItemListEnd
     \resumeSubHeadingListEnd

%-----------Workshops Taken-----------------
\section{Workshops Taken}
  \resumeSubHeadingListStart
    \resumeSubheading
      {Raspberry Pi Workshop}{CET, Thiruvananthapuram}
      {IOT Lab}{2018}
      \resumeItemListStart
        \resumeItem{Python}
          {This workshop was conducted by a team of 4 students including me. In this workshop, I handled Communication session.}
      \resumeItemListEnd
      \resumeSubheading
      {Arduino Workshop}{CET, Thiruvananthapuram}
      {IOT Lab}{2018}
      \resumeItemListStart
        \resumeItem{Arduino}
          {This workshop covered all the sections from Arduino basics, embedded C, micro controller vs micro processor, sensors and few basic programs.}
      \resumeItemListEnd
      
      \resumeSubheading
      {Beaglebone Workshop}{CET, Thiruvananthapuram}
      {IOT Lab}{2018}
      \resumeItemListStart
        \resumeItem{Hardware}
          {This workshop was conducted by a team of 4 students including me. In this workshop, I handled hardware session.}
      \resumeItemListEnd
      
      \resumeSubheading
      {Raspberry Pi Workshop}{CET, Thiruvananthapuram}
      {IOT}{2018}
      \resumeItemListStart
        \resumeItem{Pi}
          {This workshop was conducted at Drishti (Technical Fest, CET). The session included about introduction Raspberry Pi, Linux, working remotely, Python and home automation using Pi.}
      \resumeItemListEnd
      
  \resumeSubHeadingListEnd
  
%-----------MY PROJECTS-----------------
\section{My Projects}
  \resumeSubHeadingListStart
    \resumeSubItem{Build your own google home}
      { DIY google home smart speaker using raspberry pi. This project is available at \url{https://github.com/theonlyakhil/Build-your-own-google-home}.}
       \resumeSubItem{System Information Display Raspberry Pi}
      {An 0.96 inch OLED display is used for showing system informations of raspberry pi. This project is available at \url{https://github.com/theonlyakhil/system-information-display-for-raspberry-pi}.}
      \resumeSubItem{CNC Machine}
      {CNC machine using old DVD drive from computer with GRBL driver and arduino.}
      \resumeSubItem{Weather Display}
      {A smart weather display that shows weather forecast informations with a OLED display built in.}
      \resumeSubItem{Smart Plug}
      {A plug that can control with google assistant voice commands.}
      \resumeSubItem{Battery Management unit (BMS)}
      {A battery management system for portable embedded devices ,using the IC from TI BQ25895. This project is available at \url{https://github.com/theonlyakhil/BQ25895-smart-battery-management-circuit-for-portable-devices}}
      \resumeSubItem{Smart Switch with status memory}
      {A Switch that can be controlled via both physically or through mobile phone.}
      \resumeSubItem{Open CV}
      {Object detection and object tracking using openCV,for my eyanthra project .}
      
  \resumeSubHeadingListEnd

%--------PROGRAMMING SKILLS------------
\section{Programming Skills}
  \resumeSubHeadingListStart
    \item{
      \textbf{Languages}{: Embedded C, Python,C++}
      \hfill
      \textbf{Technologies}{: Linux, TKinter}
    }
    \item{
      \textbf{Softwares}{: Atmel Studio, Eagle, Fritzing, KiCAD, EasyEDA, Arduino IDE,Android Studio,Visual Studio code}
    }

  \resumeSubHeadingListEnd
  
%-------------------------------------------
\end{document}
